\documentclass{article}
\title{\textbf{Gravitation Formula}}
\date{June 2022}
\author{Damyanti NA21B020}

\usepackage{graphicx}


\begin{document}
  \maketitle
\large
\paragraph{} \textbf{Gravity or Gravitation Formula} is a force that occurs among all material objects in the universe. For any two objects or units having non-zero mass, the force of gravity has a tendency to attract them toward each other. Newton’s Law of Universal Gravitation states that:

“Every particle attracts every other particle in the universe with force directly proportional to the product of the masses and inversely proportional to the square of the distance between them”.

If the distance between two masses $m_{1}$ and $m_{2}$ is r, then the gravity formula is articulated as:

\vspace{1cm}


\boldmath
\begin{equation}
  F=G\frac{{m_1}{m_2}}{{r^2}}
\end{equation}

\vspace{1cm}

\begin{tabular}{|c|l|}
    \hline
    $G$ & a constant equal to 6.67 × 10-11 N.m^{2}.kg^{-2}\\
    $m_{1}$ & the mass of the body 1\\
    $m_{2}$ & the mass of body 2\\
    $r$ & the radius or distance between the two bodies\\
    \hline
\end{tabular}


\end{document}
